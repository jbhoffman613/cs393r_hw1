\section{Question 1: Parameters} 
Our algorithm required the wheel base distance, the width of the car, the length of the car, the current curvature, the current velocity, the latency, the number of proposed curvatures to consider at any given time, the max curvature value, and the front safety margin, and a side safety margin. Noticeably missing, we did not use the track distance of the car. 

Additionally, to execute the J-Turn maneuver, we have a variable titled $FLRR\_DURATION$ used to describe what state we are in for the J-Turn, we have the J-Turn velocity parameter, and the J-Turn curvature value which are the present values to turn along and the speed to drive at when executing the maneuver. 

All of the length and width measurements are used in calculating the free-path for a specific arc. They are also used when checking if the car will collide with any point along a proposed curvature, and also used when deciding if a point will collide with the front or the side of the car. The current velocity is used to determine what the next velocity should be based on the calculated free path length in our 1D TOC method. The front of side safety margins while implicitly included in the length and width of our car, have an impact in calculating the clearance and also it impacts the free path length calculations which in turn impacts our 1D TOC calculations thus ensuring that we do not collide with a wall upon breaking. We also used the current velocity, the latency parameter, and the current curvature to forward predict the motion of the car during latency prediction and when updating the point cloud. Finally, the curvature number and the max curvature value represent the number of curves to consider in between the highest and lowest allowed curvature values, and the tightest curve that we will allow the car to take either left or right respectively. 